\section{Introduction}
Introduction of the document.

\subsection{Scope}

Abgabe-Teil Openstack \\
by Peter Buzanits - Friday, 1 April 2022, 7:25 PM \\
Number of replies: 0 \\

\noindent
Für jene, die Anfang März nicht dabei waren nochmal die Aufgabenstellung für
die Abgabe für den Openstack-Teil der PT InEn: \\

\noindent
Es soll auf der Openstack-Installation der FH, auf der wir bereits geübt
haben, eine Infrastruktur mit eigenem/n Netzwerk/en und mehreren Instanzen
erstellt werden. Darauf soll ein oder mehrere Services laufen. Idealerweise
stehen diese Services in Zusammenhang mit dem Thema der Arbeit, die für die
InEn-ILV erstellt wird. Deshalb sind auch die Abgabegruppen die selben wie in
der ILV. \\

\noindent
So könnte z. B. irgend etwas demonstriert werden, dass in der Arbeit
beschrieben wird. Oder es werden die Ergebnisse in einem Wiki präsentiert oder
ähnliches. Wenn das nur schwer möglich ist, kann das aber auch ganz
unabhängig von der ILV sein. \\

\noindent
Vorsicht! Alles für die Abgabe soll im Projekt InEn-GruppeX stattfinden (X ist
hier Platzhalter für den Gruppennamen). Das Projekt InEn-21107810xx steht
weiterhin für persönliche Experimente bis zum Ende des Semesters zur
Verfügung. \\

\noindent
Weiters soll ein Bericht abgegeben werden, in dem die Infrastruktur beschrieben
ist. Das soll so gestaltet sein, dass jemand, der diese Infrastruktur
übernimmt (z. B. als Nachfolger des Admins in einem Unternehmen) diese ohne
weitere Nachfragen verwalten kann. Es sollen also alle Aspekte wie Netzwerke,
IP-Adressen, Securitygroups etc. beschrieben sein. Auch die Services sollen
nach Lesen dieses Berichtes benutzt werden können. \\

\noindent
Es wird eine Draft-Abgabe des Berichtes geben, die schon den kompletten Umfang
haben soll. Damit kann ich vor der Endabgabe noch Feedback geben. Die Services
müssen zum Zeitpunkt des Drafts noch nicht funktionieren. \\

\noindent
Wer bei der letzten Übung nicht da war und seine Login-Credentials nicht
weiß, bitte mir eine E-Mail schicken. \\

