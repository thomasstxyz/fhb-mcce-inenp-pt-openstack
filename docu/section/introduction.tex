\section{Introduction}
In the introduction, the given task and the chosen approach are described.

\subsection{Scope}

Openstack Assignment by Peter Buzanits (translated from German): \\

\noindent
On the Openstack installation of the University of Applied Sciences 
Burgenland, on which we have already practised
infrastructure with its own network(s) and multiple instances.
One or more services should run on it. Ideally
these services are related to the topic of the work that is created for the
InEn-ILV. Therefore, the delivery groups are the same as in the
the ILV. \\

\noindent
For example, something could be demonstrated that is described in the work.
Or the results are presented in a wiki or similar.
If this is difficult to do, it can also be done independently of the
of the ILV. \\

\noindent
Caution. Everything for the submission should take place in the project InEn-GruppeX (X is
placeholder for the group name here). The project InEn-21107810xx is still available
for personal experiments until the end of the semester. \\

\noindent
Furthermore, a report is to be submitted in which the infrastructure is described.
This should be designed in such a way that someone who takes over this infrastructure
infrastructure (e.g. as a successor to the administrator in a company) can manage it without
further inquiries. So all aspects like networks,
IP addresses, security groups, etc. should be described. Also the services should
be accessible for use after reading this report. \\

\noindent
There will be a draft version of the report, which should already have the complete scope.
This will allow me to provide feedback before the final submission. The services
do not have to be working at the time of the draft. \\

\noindent
If you were not there for the last exercise and do not know your login credentials,
please send me an email.

\subsection{Approach}

For the given task - setting up infrastructure on 
an Openstack \cite{openstackWebsite} instance provided by the university -
the chosen approach was to define the infrastructure and all configuration
as code by the concept of Infrastructure-as-Code (IaC) \cite{iacDefinitionIBM}.
To achieve this, the following open source software tools are used:

\begin{itemize}
	\item Hashicorp Terraform \cite{terraformWebsite}
	\item Red Hat Ansible \cite{ansibleWebsite}
\end{itemize}

\noindent
Terraform is used to execute plans written in the 
Hashicorp Configuration Language \cite{hclWebsite},
which is a superset of the common JSON configuration language.
By nature, it is a declarative language, which means that
you define a desired end result, instead of writing imperative statements.
In this way, the code can achieve the feature of idempotency,
which means that it can be applied multiple times without 
changing the result beyond the initial application. \\

\noindent
These Terraform plans define all infrastructure components
within Openstack like Network, Routers, Subnets, Floating IPs, Instances, etc.,
and their parameters like network CIDRs, IP addresses, vCPU count, disk and memory space etc. \\

\noindent
Terraform plans can be applied, upon the Terraform engine will make
the according HTTP calls to the Openstack API, which will result in the infrastructure
being scheduled for provisioning. \\

\noindent
After successful provisioning of the infrastructure resources on Openstack,
Terraform will return the newly created Floating IP addresses, 
which are associated with the compute instances. \\

\noindent
These IP addresses are written
into the Ansible inventory \cite{ansibleInventoryWebsite}, which is a configuration
file in the ini-style used by Ansible. In the Ansible inventory, the hosts, which 
Ansible should target, are defined, as well as additional variables if needed,
like which user to connect with or which port the SSH daemon listens at.
For connecting to the hosts, Ansible uses simple SSH connections.
Ansible uses so called Ansible Playbooks \cite{ansiblePlaybooksWebsite}
to define the configuration, which are written in the YAML syntax. \\

\noindent
Ansible Playbooks are made up of many small tasks, which most of the time basically
represent shell commands, with the difference, that they are defined as YAML,
which has the advantage of being easy to read and understand even for non-programmers.
Additionally, Ansible Playbooks also declarative by nature, which means a desired end result 
is defined. Ansible then transparently takes care of achieving the desired outcome,
no matter the starting point, all while keeping the code base as simple as possible.
With Ansible Playbooks being very easy to read, it is often times not needed to write 
more sophisticated documentation.
Ansible excels at automating on the operating system layer,
which means running system updates, installing additional software,
configuring. All this automation can be done on multiple hosts, without 
having to login to any host by hand. 

\subsection{Result}
The end result after applying this infrastructure code is a 
Kubernetes Cluster consisting of one master (control plane) and two worker nodes.
The virtual machine instances are provisioned via
Terraform on Openstack. IP addresses of the machines are handed off to Ansible, which continues
the provisioning process at the operating system level
after Terraform successfully starts the Ubuntu 20.04 machine images.
Ansible connects via the SSH key pair, which is provided to 
Terraform prior to \verb|terraform apply| by the operator.
Ansible updates the base system packages, installs necessary
components for Kubernetes like \verb|kubectl| and overlay network,
and finally create a Kubernetes cluster and joining all nodes together.
The root user on the master (control plane) machine will have access to
kubeadm and kubectl.




