\section{Introduction}
In the introduction, the given task and the chosen approach are described.

\subsection{Scope}

Abgabe-Teil Openstack \\
by Peter Buzanits - Friday, 1 April 2022, 7:25 PM \\
Number of replies: 0 \\

\noindent
Für jene, die Anfang März nicht dabei waren nochmal die Aufgabenstellung für
die Abgabe für den Openstack-Teil der PT InEn: \\

\noindent
Es soll auf der Openstack-Installation der FH, auf der wir bereits geübt
haben, eine Infrastruktur mit eigenem/n Netzwerk/en und mehreren Instanzen
erstellt werden. Darauf soll ein oder mehrere Services laufen. Idealerweise
stehen diese Services in Zusammenhang mit dem Thema der Arbeit, die für die
InEn-ILV erstellt wird. Deshalb sind auch die Abgabegruppen die selben wie in
der ILV. \\

\noindent
So könnte z. B. irgend etwas demonstriert werden, dass in der Arbeit
beschrieben wird. Oder es werden die Ergebnisse in einem Wiki präsentiert oder
ähnliches. Wenn das nur schwer möglich ist, kann das aber auch ganz
unabhängig von der ILV sein. \\

\noindent
Vorsicht! Alles für die Abgabe soll im Projekt InEn-GruppeX stattfinden (X ist
hier Platzhalter für den Gruppennamen). Das Projekt InEn-21107810xx steht
weiterhin für persönliche Experimente bis zum Ende des Semesters zur
Verfügung. \\

\noindent
Weiters soll ein Bericht abgegeben werden, in dem die Infrastruktur beschrieben
ist. Das soll so gestaltet sein, dass jemand, der diese Infrastruktur
übernimmt (z. B. als Nachfolger des Admins in einem Unternehmen) diese ohne
weitere Nachfragen verwalten kann. Es sollen also alle Aspekte wie Netzwerke,
IP-Adressen, Securitygroups etc. beschrieben sein. Auch die Services sollen
nach Lesen dieses Berichtes benutzt werden können. \\

\noindent
Es wird eine Draft-Abgabe des Berichtes geben, die schon den kompletten Umfang
haben soll. Damit kann ich vor der Endabgabe noch Feedback geben. Die Services
müssen zum Zeitpunkt des Drafts noch nicht funktionieren. \\

\noindent
Wer bei der letzten Übung nicht da war und seine Login-Credentials nicht
weiß, bitte mir eine E-Mail schicken. \\

\subsection{Approach}

For the given task - setting up infrastructure on 
an Openstack \cite{openstackWebsite} instance provided by the university -
the chosen approach was to define the infrastructure and all configuration
as code by the concept of Infrastructure-as-Code (IaC) \cite{iacDefinitionIBM}.
To achieve this, the following open source software tools are used:

\begin{itemize}
	\item Hashicorp Terraform \cite{terraformWebsite}
	\item Red Hat Ansible \cite{ansibleWebsite}
\end{itemize}

\noindent
Terraform is used to execute plans written in the 
Hashicorp Configuration Language \cite{hclWebsite},
which is a superset of the common JSON configuration language.
By nature, it is a declarative language, which means that
you define a desired end result, instead of writing imperative statements.
In this way, the code can achieve the feature of idempotency,
which means that it can be applied multiple times without 
changing the result beyond the initial application. \\

\noindent
These Terraform plans define all infrastructure components
within Openstack like Network, Routers, Subnets, Floating IPs, Instances, etc.,
and their parameters like network CIDRs, IP addresses, vCPU count, disk and memory space etc. \\

\noindent
Terraform plans can be applied, upon the Terraform engine will make
the according HTTP calls to the Openstack API, which will result in the infrastructure
being scheduled for provisioning. \\

\noindent
After successful provisioning of the infrastructure resources on Openstack,
Terraform will return the newly created Floating IP addresses, 
which are associated with the compute instances. \\

\noindent
These IP addresses are written
into the Ansible inventory \cite{ansibleInventoryWebsite}, which is a configuration
file in the ini-style used by Ansible. In the Ansible inventory, the hosts, which 
Ansible should target, are defined, as well as additional variables if needed,
like which user to connect with or which port the SSH daemon listens at.
For connecting to the hosts, Ansible uses simple SSH connections.
Ansible uses so called Ansible Playbooks \cite{ansiblePlaybooksWebsite}
to define the configuration, which are written in the YAML syntax. \\

\noindent
Ansible Playbooks are made up of many small tasks, which most of the time basically
represent shell commands, with the difference, that they are defined as YAML,
which has the advantage of being easy to read and understand even for non-programmers.
Additionally, Ansible Playbooks also declarative by nature, which means a desired end result 
is defined. Ansible then transparently takes care of achieving the desired outcome,
no matter the starting point, all while keeping the code base as simple as possible.
With Ansible Playbooks being very easy to read, it is often times not needed to write 
more sophisticated documentation.
Ansible excels at automating on the operating system layer,
which means running system updates, installing additional software,
configuring. All this automation can be done on multiple hosts, without 
having to login to any host by hand. 





