\section{Infrastructure Setup}
Explain the Terraform plans and Ansible Playbooks roughly. \\

\noindent Give an overview of the infrastructure code..

\subsection{Infrastructure provisioning}

\subsubsection{Terraform execution environment}
To get started, the Git Repository needs to be cloned.

\begin{verbatim}
	$ git clone https://github.com/thomasstxyz/fhb-mcce-inenp-pt-openstack
\end{verbatim}

\noindent
Next, the credentials for authentication to the Openstack API,
as well as the public ssh key will be
read into environment variables.

\begin{verbatim}
	$ export OS_USERNAME=my_user
	$ export OS_PASSWORD=my_pass
	$ export OS_PROJECT_ID=1234...
	$ export TF_VAR_public_key="ssh-rsa ..."	
\end{verbatim}

\noindent
The project id can be retrieved from the Openstack Horizon Dashboard
under "Project - API Access" 
\url{http://172.20.41.1/horizon/project/api_access/},
upon pressing the button "View Credentials". \\

\noindent
Now the setup is complete and Terraform should be able to authenticate
to the Openstack API. Change into the terraform directory.

\begin{verbatim}
	$ cd terraform
\end{verbatim}

\noindent
Use Terraform plan to look at what changes would be made by executing
(this is just a dry run and does not make any changes).

\begin{verbatim}
	$ terraform plan
\end{verbatim}

\noindent
A summary will be printed in the console, stating 
how many resources to add, change and destroy. \\

\begin{verbatim}
	Plan: 21 to add, 0 to change, 0 to destroy.
\end{verbatim}

\noindent
If content with the dry run, the plan can be applied.

\begin{verbatim}
	$ terraform apply
\end{verbatim}

\noindent
Again a summary will be printed, and the user is asked to confirm with yes. \\

\noindent
After a successful Terraform apply, the IP addresses of the created compute 
instances are displayed on screen. Additionally, the addresses are written
to the Ansible inventory file \verb|../ansible/inventory|,
which is needed by Ansible for the next step. \\

\noindent
All resources can be deleted, by running Terraform destroy.

\begin{verbatim}
	$ terraform destroy
\end{verbatim}

\subsection{Ansible - K8s cluster \& OS configuration}

\section{Openstack resources}
List all Openstack resources, which are provisioned by Terraform.

\section{Operating Manual}
Some commands for servicing the infrastructure, updating, patching... \\

\noindent Commands for troubleshooting, see logs, ...
